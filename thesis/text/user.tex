\chapter{Felhasználói dokumentáció}
\label{ch:user}

A program elsődleges felhasználói fejlesztők, alapszintű tudás szükséges a Java nyelvről (vagy bármilyen olyan nyelvről amely JVM Bytecode-ra fordul), a class fájlokról, illetve Java programok fordításáról.

Mivel az elkészített program csak interpretálni tud, a fordítást egy már elérhető Java fordítóprogrammal szükséges megtenni. Mivel a Java programok class fájlokra fordulnak, ezek futtatásához szükséges egy interpretáló program.

Alapvető esetben ez a fordítóprogrammal együtt telepítésre kerül. A szakdolgozat esetében a lefordított class fájl futtatásával képesek lehetünk más, már lefordított Java programot futtatni.

A mellékelt fájlok között elérhető egy jar fájl is, ennek a futtatásához ugyanúgy szükségünk van egy beépített interpretáló programra, amely képes Java programokat futtatni és nem a szakdolgozat maga.

\section{Kikötések}
A program csak Java 7-nél újabb fordítóprogrammal fordított Java programokat képes interpretálni, számos Bytecode instrukciót a Java 7-es verziójában elavulttá tettek (ezek: \lstinline{ret}, \lstinline{jsr}, \lstinline{jsr_w}), nem fordulnak elő class fájlokban. A szakdolgozat ezeket az instrukciókat nem implementálta.

Ezen felül egy másik instrukció is implementálatlan maradt (\lstinline{invokedynamic}), tehát nem minden program futtatható. Ha a class fájlok egyike tartalmazza ezt az instrukciót, akkor a program jelez a felhasználó számára. Akaratlanul is része lehet a programunknak ez az instrukció, amikor egy változót szöveggel együtt próbálunk kiírni:
\begin{minted}{java}
String world = "world";
System.out.println("Hello " + world);
\end{minted}
akkor a legtöbb fordítóprogram egy \lstinline{invokedynamic} utasítást is elhelyez a programunkban.

Ez viszont elkerülhető, ha megfelelő flagekkel fordítjuk le a programunkat, mégpedig a \lstinline{-XDstringConcat=inline} flag használatával az \lstinline{invokedynamic} nem fog szerepelni a string konkatenációnál.

\section{Fordítástól futásig}

\subsection{Minimum követelmények}

A program fordításához legalább a Java 17-es verziója szükséges. Ez alatt a program fordulni sem képes, mivel pár olyan funkciót használ, amely csak a 17-es verzióban lett bevezetve.

A könnyebb fordítás (illetve egyszerűbb jar fájl készítés) érdekében a Maven fordítás automatizálási program telepítése ajánlott, ezen belül is a 3.9.0-ás verzió.

\subsection{Fordítás}

Ha nem akarunk Maven-t használni, akkor a fordítás menete a következő:
\begin{compactitem}
	\item Menjünk a \lstinline{src/main/java} mappába: \lstinline{cd src/main/java/}
	\item Fordítsuk le a \lstinline{com/zoltanbalazs/Main.java} fájlt: \lstinline{javac com/zoltanbalazs/Main.java}
	\item Az elkészült class fájl a \lstinline{src/main/java/com/zoltanbalazs} mappában lesz 
\end{compactitem}

Maven-t használva ez a procedúra egyszerűbb:
\begin{compactitem}
	\item Futtassuk le a csomagoló parancsot: \lstinline{mvn package}
	\item Az elkészült class fájl a \lstinline{target/classes/com/zoltanbalazs} mappában lesz, ezen felül a \lstinline{target} mappában lesz egy futtatható jar fájl is
\end{compactitem}

\subsection{Futtattás}

Ha a generált class fájl-lal akarjuk futtatni a programot, futtassuk le a \lstinline{java com.zoltanbalazs.Main} parancsot a \lstinline{src/main/java} mappában.\ (ha Maven-nel fordítottunk akkor a \lstinline{target/classes} mappában futtassuk le az előző fenti parancsot)

A maven által készített jar fájl-lal való futtatáshoz, futtassuk le a \lstinline{java -jar target/jabyinja-1.0.0.jar} parancsot a főmappában.

Mindkét esetben egy opcionális argumentumot (argumentum sorozatot ha a futtatandó programunk vár parancssori argumentumot) meg tudunk adni, ez a \lstinline{main} metódust tartalmazó class fájl elérési útvonala. Alapvető esetben a program a futási mappában próbál meg egy \lstinline{Main.class} fájlt futtatni.

Futásra egy példa:
\lstinline{java -jar target/jabyinja-1.0.0.jar target/test-classes/com/zoltanbalazs/PTI/_01/Greet.class World}

\subsection{Önfuttatás}

Az elkészült interpreter képes saját magát is futtatni, ehhez a futtatáshoz hasonlóan meg kell adni a programnak a saját class fájlának elérési útját, majd opcionálisan a többi paramétert.

Ez jar fájl esetén így néz ki, a főkönyvtárból futtatva:
\lstinline{java -jar target/jabyinja-1.0.0.jar target/classes/com/zoltanbalazs/Main.class target/test-classes/com/zoltanbalazs/PTI/_01/Greet.class World}

\section{Felmerülő problémák}

A futtatandó program futása során nem merül fel probléma (hacsak nincsen \lstinline{invokedynamic} a generált class fájlban) amelyet a program okoz. Ha a futtatandó programunk hibát dob, akkor ezt az interpretáló program is ugyanúgy megteszi; viszont a hiba kiírása során nem biztos hogy ugyanazt a kimentet kapjuk mint a beépített interpreter-rel.

Tehát ha a hibánk nem egy \lstinline{try, catch} blokk-ban szerepel, akkor a kiírt üzenet nem biztos hogy ugyanaz lesz mint a beépített interpreter-rel, az összes többi kiírt üzenet viszont ugyanaz kell hogy legyen.