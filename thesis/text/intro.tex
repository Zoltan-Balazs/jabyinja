\chapter{Bevezetés}
\label{ch:intro}

\section{Motiváció}

Kifejezetten érdekel a számítógép hardveréhez közeli problémák megértése és megoldása. C programozói háttérel rendelkezve az x86-64, illetve ARM assembly-ben való programozástól sem rémülök meg. Viszont amennyire azt tudom hogy azon assembly nyelvek hogyan működnek (regiszterek, \lstinline{stack}, szubrutinok, ciklusok, feltételek), annyira nem tudom hogy egy magasabb, Objektumorientált programozási nyelv (azon belül is egy fordított és interpretált nyelv) hogyan "kommunikál a géppel".

A Java bájtkód egy tökéletes köztes állapot, míg ha egy C++ interpretert akarok írni akkor valójában egy fordítóprogramot kell írnom, addig Java nyelvnél elegendő a köztes nyelv interpretálását megvalósítani, a fordítást pedig a beépített fordítóprogram hagyni.

\section{Leírás}

A Java nyelvben írt programok fordításukat követően nem egy közvetlen futtatható állományra (gépi kódra) fordulnak (a fordítást általában a beépített \lstinline{javac} program végzi el), hanem egy köztes nyelvre, Java bájtkódra, amelyet aztán különböző programokkal az adott architektúrán interpretáljuk. Legtöbb esetben az interpretálást a JVM (Java Virtual Machine, magyarul Java virtuális gép) interpretere hajtja végre (ez a beépített \lstinline{java} program).

A szakdolgozat célja egy kiegészítő program (fantázianevén \textit{Jabyinja}: \textit{\textbf{Ja}va \textbf{by}tecode \textbf{in}terpreter in \textbf{Ja}va}) írása, amely ugyan hagyatkozik a \lstinline{javac} és \lstinline{java} programokra (az előbbire a fordítás, az utóbbira a futtatás miatt), de a tényleges futtatást a különböző Java bájtkód instrukciók (utasítások) implementálásval végzi el.

A program nincsen Java kód interpretálásához kötve, a Java bájtkód a neve ellenére nem csak a Java programozási nyelvnek a bájtkódja, más programozási nyelveknek is az alapja (ezek közül pár: Clojure, Kotlin, Scala), pontosabban azoknak amelyek a JVM-et használják fel, viszont a tesztelés csak Java kódból generált Java bájtkódra tér ki, ugyanis a szakdolgozat céljaként az ELTE Programtervező Informatikus BSc szakán elkészített Java programok lefordított class fájlainak interpretálását tűztem ki.

A programnak szükséges értelmeznie kell egy adott class fájlt (többet is ha egy külön fájlra is hivatkozunk), helyesen beolvasnia a benne lévő adatokat, majd a belépési (\textit{main}) metódust lefuttatnia. A program erősen alapszik a Java nyelvbe beépített reflekcióra, ezen felül saját stack és lokális változók implementálása is szükséges. Mivel a Java nyelvre épül a program, ezért saját heap megírására nincsen szükség, ez automatikusan kezelve lesz a standard Java interpreter által.

A program a JVM specifikációt\cite{jvmspecification} (ennek is a 7-es verizóját) követi, azt megpróbálja tökéletesen implementálni, minden egyes megoldási döntésével (jó vagy rossz) együtt.