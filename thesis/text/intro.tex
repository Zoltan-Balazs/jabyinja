\chapter{Bevezetés}
\label{ch:intro}

A Java nyelvben írt programok fordításukat követően nem egy közletlen futtatható állományra (gépi kódra) fordulnak (a fordítást általában a beépített \lstinline{javac} program végzi el), hanem egy köztes nyelvre, bytecode-ra, amelyet aztán különböző programokkal az adott architektúrán interpretáljuk. Legtöbb esetben az interpretálást a JVM (Java Virtual Machine) interpretere hajtja végre (ez a beépített \lstinline{java} program).

A szakdolgozat célja egy kiegészítő program (fantázianevén \textit{Jabyinja} $-$ \textit{\textbf{Ja}va \textbf{by}tecode \textbf{in}terpreter in \textbf{Ja}va}) írása, amely ugyan hagyatkozik a \lstinline{javac} és \lstinline{java} programokra (az előbbire a fordítás, az utóbbira a futtatás miatt), de a tényleges futtatást a különböző bytecode instrukciók implementálásval végzi el.

A program nincsen Java kód interpretálásához kötve, a Java bytecode a neve ellenére más programozási nyelveknek is az alapja (ezek közül az ismertebbek: Kotlin, Clojure), viszont a tesztelés csak Java kódból generált bytecodera tér ki, ugyanis a szakdolgozat céljaként az ELTE Programtervező Informatikus BSc szakán elkészített Java programok fordításának interpretálását tűztem ki.

A programnak szükséges értelmeznie kell egy adott Classfájlt (többet is ha egy külön fájlra is hivatkozunk), helyesen beolvasnia a benne lévő adatokat, majd a belépési (\textit{main}) metódust lefuttatnia. A program erősen alapszik a Java nyelvbe beépített reflekcióra, ezen felül saját stack implementálása is szükséges. Mivel a Java nyelvre épül a program, ezért saját heap megírására nincsen szükség, ez automatikusan kezelve lesz.