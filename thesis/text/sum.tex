\chapter{Összegzés}
\label{ch:sum}

A szakdolgozat célja egy Java bájtkód interpret írása volt. Habár a szakdolgozat témáját nem sikerült teljes egészében lefedni (\lstinline{invokedynamic} utasítás), ennek ellenére elképesztően hasznos volt számomra a téma során végrehajtott kutatás. Alkalmam nyílt a JVM belső működését jobban megérteni, beleértve a class fájl felépítését.

Habár a kód biztosan nem a legszebb megoldásokat nyújtja a reflekció használata miatt, de jelenlegi tudásom szerint megfelelően van felépítve, és más számára is olvasható.

Kézzel nem fogok class fájlokat, se bájtkódot, írni, de ha valaha látok Java bájtkódot, akkor tudom mely utasítás pontosan mit is csinál, és ezen túl azt is tudni fogom, hogy merre keressem ha valamely utasítás paramétere nem tudnám hogy mi (értem itt ha a \lstinline{Constant Pool}-ban lévő elemre mutat egy utasítás).

A \lstinline{javap} program, amely a Java SDK-val jön (ennek a létezésére nem is tudtam ezelőtt) elképesztően hasznos volt számomra mind a szakdolgozat megírása során, és ezentúl is szerintem használni fogom.