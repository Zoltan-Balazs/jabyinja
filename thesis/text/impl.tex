\chapter{Fejlesztői dokumentáció}
\label{ch:impl}

\section{Class fájl felépítése}

\subsection{Class fájltól a benne levő metódus futtatásáig}

A fő osztály a \lstinline{ClassFile}, ez felel számos dologért, többek között egy class fájl beolvasáért, a megfelelő adattagok beállításával. A \lstinline{ClassFile} osztálynak egy konstruktora van, mégpedig:
\begin{minted}{java}
public ClassFile(String fileName, String[] mainArgs)
\end{minted}
Tehát az első paraméter a beolvasandó class fájl neve, a második pedig a \lstinline{main} metódusnak adott argumentumok.

Az implementáció alapján nem kötött a \lstinline{main} metódus használata belépési pontként, tehát a 2.\ argumentum lehet \lstinline{null} is.

A konstruktor meghívása egyidejüleg meghívja a \lstinline{readClassFile} függvényt is:
\begin{minted}{java}
public void readClassFile(String fileName)
\end{minted}
Ez a függvény egy adott fájlnévre beolvassa a class fájlban tárolt adatokat megfelelő változókba.
(Ezen felül egy \lstinline{VALID_CLASS_FILE} változót is beállít; feltétellezük hogy ha a mágikus szám \lstinline{(CA FE BA BE)} megtalálható a fájl elején, akkor az adott fájl egy valid class fájl, ellenkező esetben egy \lstinline{InvalidClassFileException}-t dob a beolvasó függvény.)

A beolvasás után (tehát az objektum létrehozása után) érdemes a belépési függvényt (általában \lstinline{main}) megkeresni a \lstinline{findMethodsByName} metódussal:
\begin{minted}{java}
public Method_Info findMethodsByName(String methodName)
\end{minted}
Ez egy adott függvénynévre a megfelelő nevü metódust visszaadja a beolvasott fájlból (ha nem talál ilyet akkor \lstinline{null}-t ad vissza).
Egy példa a használatára:
\begin{minted}{java}
ClassFile CLASS_FILE = new ClassFile("Main.class", null);
Method_Info method = CLASS_FILE.findMethodsByName("main");
\end{minted}

A függvény megtalálása után ajánlott a \lstinline{Code} attribútumot megtalálni, ebben, többek között, található a futtatandó bytecode is. A segédfüggvény erre a \lstinline{findAttributesByName}:
\begin{minted}{java}
public List<Attribute_Info> findAttributesByName(List<Attribute_Info> attributes, String attributeName)
\end{minted}
Mivel egy attribútumból több is lehet, egy listát kapunk vissza (a \lstinline{Code}-ból csak egy lesz), bemeneti paraméterként az attribútumnév mellett a megfelelü függvény attribútumait is át kell adnuk, például:
\begin{minted}{java}
List<Attribute_Info> attributes = CLASS_FILE.findAttributesByName(method.attributes, "Code");
\end{minted}
(Ha nem talál ilyen nevezetű attribútumot akkor üres listát ad vissza.)

A megfelelüen beolvasott attribútum után, a megtalált attribútumok között ajanlott végigmenni, a \lstinline{List} implementálja az \lstinline{Iterable}-t, így egy for ciklussal elegánsan megtehetjük ezt:
\begin{minted}{java}
for (Attribute_Info attribute : attributes)
\end{minted}

Mivel \lstinline{Code} attribútumokról beszélünk, ezért a következő ajánlott dolog hogy ebből az attribútumból olvassuk be az adatokat. Ehhez a \lstinline{Code_Attribute_Helper} osztály \lstinline{readCodeAttributes} metódusa megfelelő:
\begin{minted}{java}
public static Code_Attribute readCodeAttributes(Attribute_Info attribute) throws IOException
\end{minted}
A függvény egy attribútumot vár (például az előbbi kódrészlet \lstinline{attribute} változóját), majd pedig beolvassa a specifikációnak megfelelően a \lstinline{Code_Attribute}-ot, és visszaadja azt, ha valamiért nem sikerült a beolvasás akkor \lstinline{IOException}-t dob a függvény.
\begin{minted}{java}
Code_Attribute codeAttribute = Code_Attribute_Helper.readCodeAttributes(attribute);
\end{minted}

Ezt a beolvasott attribútumot a \lstinline{ClassFile} osztály fel tudja használni az \lstinline{executeCode} metódusával, mely egy \lstinline{byte[]} változót vár bemeneti paraméterként, ami a \lstinline{Code_Attribute} része:
\begin{minted}{java}
public Pair<Class<?>, Object> executeCode(byte[] code)
	throws IOException, ClassNotFoundException, NoSuchFieldException, IllegalAccessException,
	NoSuchMethodException, SecurityException, InstantiationException, IllegalArgumentException,
	InvocationTargetException, Throwable
\end{minted}
A reflekció miatt számos hibát dob vissza a függvény, ha nem helyes a kód formátuma akkor \lstinline{IOException}-t dob a függvény, a \lstinline{Throwable} az \lstinline{ATHROW} bytecode instrukció miatt szükséges (ekkor egy hibát dob vissza a metódusunk).
Visszatérési értéke \lstinline{Pair<Class<?>, Object>}, a számos \lstinline{RETURN} utasítás miatt (ezeket a stack-en szükséges elhelyezünk)
Példa a használatára:
\begin{minted}{java}
CLASS_FILE.executeCode(codeAttribute.code);
\end{minted}

Ezzel el is jutottunk egy class fájl beolvasásától, az abban lévő adott függvény bytecodejának futtatásáig, több teendőnk nincsen, a program az adott függvényben levő külön függvényhívásokat automatikusan elvégzi.

A teljes példakód:
\begin{minted}{java}
ClassFile CLASS_FILE = new ClassFile("Main.class", null);
Method_Info method = CLASS_FILE.findMethodsByName("main");
List<Attribute_Info> attributes = CLASS_FILE.findAttributesByName(method.attributes, "Code");

for (Attribute_Info attribute : attributes) {
	Code_Attribute codeAttribute = Code_Attribute_Helper.readCodeAttributes(attribute);
	CLASS_FILE.executeCode(codeAttribute.code);
}
\end{minted}

\subsection{Pár minta class fájl felépítése}

A legegyszerűbb class fájl ami értelmes, viszont nem futattható:

\begin{center}
\begin{tabular}{ c c c c c c c c c c c c c c c }
\stagemagic{CA} & \stagemagic{FE} & \stagemagic{BA} & \stagemagic{BE} & \stageminor{00} & \stageminor{00} & \stagemajor{00} & \stagemajor{00} & \stageconstantsize{00} & \stageconstantsize{00} & \stageaccessflags{00} & \stageaccessflags{00} & \stagethisclass{00} & \stagethisclass{00} & \stagesuperclass{00} \\
\stagesuperclass{00} & \stageinterfacesize{00} & \stageinterfacesize{00} & \stagefieldsize{00} & \stagefieldsize{00} & \stagemethodsize{00} & \stagemethodsize{00} & \stageattributes{00} & \stageattributes{00}
\end{tabular}
\end{center}

Java kódban ennek megfelelője az üres fájl:
\begin{minted}{java}
\end{minted}

Class fájl formátumának magyarázata:

\begin{compactitem}
\setlength\itemsep{-5px}
\item \stagemagic{CA FE BA BE}: Mágikus szám, amely minden Class fájl elején megtalálható
\item \stageminor{00 00} \stagemajor{00 00}: Class fájl \lstinline{Minor} és \lstinline{Major} verziószáma, egy táblázatnak megfelelően a fordítóprogram verziója
\item \stageconstantsize{00 00}: A Constant Pool mérete (+1, mivel 1-től indexelt, itt nem számít)
\item \stageaccessflags{00 00}: Hozzáférési zászlók ()
\item \stagethisclass{00 00}: \lstinline{This} osztály indexe a Constant Pool-ban
\item \stagesuperclass{00 00}: \lstinline{Super} osztály indexe a Constant Pool-ban
\item \stageinterfacesize{00 00}: Interfészek száma
\item \stagefieldsize{00 00}: Adattagok száma
\item \stagemethodsize{00 00}: Függvények száma
\item \stageattributes{00 00}: Osztály attribútumainak száma
\end{compactitem}

\pagebreak

A legegyszerűbb class fájl amit a $Jabyinja$ program le tud futtatni (a beépített \lstinline{java} program nem képes ezt lefuttatni, mivel nincsenek benne osztályok, a JVM specifikáció alapján az osztályok elhanyagolhatóak):

\begin{center}
\begin{tabular}{ c c c c c c c c c c c c c c c }
\stagemagic{CA} & \stagemagic{FE} & \stagemagic{BA} & \stagemagic{BE} & \stageminor{00} & \stageminor{00} & \stagemajor{00} & \stagemajor{00} & \stageconstantsize{00} & \stageconstantsize{04} & \stageconstantpool{01} & \stageconstantpool{00} & \stageconstantpool{04} & \stageconstantpool{43} & \stageconstantpool{6F} \\
\stageconstantpool{64} & \stageconstantpool{65} & \stageconstantpool{01} & \stageconstantpool{00} & \stageconstantpool{04} & \stageconstantpool{6D} & \stageconstantpool{61} & \stageconstantpool{69} & \stageconstantpool{6E} & \stageconstantpool{01} & \stageconstantpool{00} & \stageconstantpool{03} & \stageconstantpool{28} & \stageconstantpool{29} & \stageconstantpool{56} \\
\stageaccessflags{00} & \stageaccessflags{21} & \stagethisclass{00} & \stagethisclass{00} & \stagesuperclass{00} & \stagesuperclass{00} & \stageinterfacesize{00} & \stageinterfacesize{00} & \stagefieldsize{00} & \stagefieldsize{00} & \stagemethodsize{00} & \stagemethodsize{01} & \stagemethods{00} & \stagemethods{09} & \stagemethods{00} \\ 
\stagemethods{02} & \stagemethods{00} & \stagemethods{03} & \stagemethods{00} & \stagemethods{01} & \stagemethods{00} & \stagemethods{01} & \stagemethods{00} & \stagemethods{00} & \stagemethods{00} & \stagemethods{0D} & \stagemethods{00} & \stagemethods{00} & \stagemethods{00} & \stagemethods{00} \\
\stagemethods{00} & \stagemethods{00} & \stagemethods{00} & \stagemethods{01} & \stagemethods{B1} & \stagemethods{00} & \stagemethods{00} & \stagemethods{00} & \stagemethods{00} & \stageattributes{00} & \stageattributes{00}
\end{tabular}
\end{center}

Java kód megfelelője:
\begin{minted}{java}
public static void main() {
    return;
}
\end{minted}

Class fájl formátumának magyarázata:

\begin{compactitem}
\setlength\itemsep{-5px}
\item \stagemagic{CA FE BA BE}: Mágikus szám, amely minden class fájl elején megtalálható
\item \stageminor{00 00} \stagemajor{00 00}: Class fájl \lstinline{Minor} és \lstinline{Major} verziószáma, egy táblázatnak megfelelően a \lstinline{javac} fordítóprogram verziója
\item \stageconstantsize{00 04}: A Constant Pool mérete (+1, mivel 1-től indexelt)
\item \stageconstantpool{01 00 04 43 6F 64 65 01 00 04 6D 61 69 6E 01 00 03 28 29 56}: Constant Pool
\begin{compactitem}
    \setlength\itemsep{-5px}
    \item \stagemajor{01} \stageminor{00 04} \stageconstantsize{43 6F 64 65} \\
    \stagemajor{01}: Constant Pool Info érték \lstinline{(CONSTANT_Utf8)} \\
    \stageminor{00 04}: 4 hosszú \\
    \stageconstantsize{43 6F 64 65}: A \lstinline{CONSTANT_Utf8} értéke: \lstinline{Code}
    \item \stagemajor{01} \stageminor{00 04} \stageconstantsize{6D 61 69 6E} \\
    \stagemajor{01}: Constant Pool Info érték \lstinline{(CONSTANT_Utf8)} \\
    \stageminor{00 04}: 4 hosszú \\
    \stageconstantsize{6D 61 69 6E}: A \lstinline{CONSTANT_Utf8} értéke: \lstinline{main}
    \item \stagemajor{01} \stageminor{00 03} \stageconstantsize{28 29 56} \\
    \stagemajor{01}: Constant Pool Info érték \lstinline{(CONSTANT_Utf8)} \\
    \stageminor{00 03}: 3 hosszú \\
    \stageconstantsize{28 29 56}: A \lstinline{CONSTANT_Utf8} értéke: \lstinline{()V}
\end{compactitem}
\item \stageaccessflags{00 21}: Hozzáférési zászlók \lstinline{(Public, Super)} - elhanyagolhatóak ebben az esetben
\item \stagethisclass{00 00}: \lstinline{This} osztály indexe a Constant Pool-ban
\item \stagesuperclass{00 00}: \lstinline{Super} osztály indexe a Constant Pool-ban
\item \stageinterfacesize{00 00}: Interfészek száma
\item \stagefieldsize{00 00}: Adattagok száma
\item \stagemethodsize{00 01}: Függvények száma
\item \stagemethods{00 09 00 02 00 03 00 01 00 01 00 00 00 0D 00 00 00 00 00 00 00 01 B1 00 00 00 00}: Függvények
\begin{compactitem}
    \setlength\itemsep{-5px}
    \item \stagemajor{00 09} \stageminor{00 02} \stageconstantsize{00 03} \stageconstantpool{00 01} \stageaccessflags{00 01 00 00 00 0D 00 00 00 00 00 00 00 01 B1 00 00 00 00} \\
    \stagemajor{00 09}: Hozzáférési zászlók \lstinline{(Public, Static)} \\
    \stageminor{00 02}: Constant Poolban lévő indexe a függvénynek: \lstinline{main} \\
    \stageconstantsize{00 03}: Függvény leírása (bemeneti paraméterek, visszatérési érték): \lstinline{()V} \\
    \stageconstantpool{00 01}: Függvény attribútumainak száma \\
    \stageaccessflags{00 01 00 00 00 0D 00 00 00 00 00 00 00 01 B1 00 00 00 00}: Attribútumok
    \begin{compactitem}
        \setlength\itemsep{-5px}
        \item[•] \stagemajor{00 01} \stageminor{00 00 00 0D} \stageconstantsize{00 00 00 00 00 00 00 01 B1 00 00 00 00} \\
        \stagemajor{00 01}: Constant Pool-ban lévő indexe az attribútumnak: \lstinline{Code} \\
        \stageminor{00 00 00 0D}: Attribútum hossza (\lstinline{0D} = 13 bájt) \\
        \stageconstantsize{00 00 00 00 00 00 00 01 B1 00 00 00 00}: Attribútum
            \begin{compactitem}
            \setlength\itemsep{0px}
                \item[–] \stagemajor{00 00} \stageminor{00 00} \stageconstantsize{00 00 00 01} \stageconstantpool{B1} \stageaccessflags{00 00} \stagethisclass{00 00} \\
                \stagemajor{00 00}: Stack mérete  \\
                \stageminor{00 00}: Lokális változók száma \\
                \stageconstantsize{00 00 00 01}: Kód hossza \\
                \stageconstantpool{B1}: Kód \lstinline{(B1 = return)}  \\
                \stageaccessflags{00 00}: Kivételek száma \\
                \stagethisclass{00 00}: Attribútum attribútumainak száma
        \end{compactitem}
    \end{compactitem}
\end{compactitem}
\item \stageattributes{00 00}: Osztály attribútumainak száma
\end{compactitem}

\subsection{Adatszerkezetek}

A class fájlnak megfelelően a két legfontosabb adattag a \lstinline{stack} és a \lstinline{local} (lokális) változók. A különböző instrukciók az ezeken lévő adatokkal dolgoznak, erre/ebbe helyeznek el megfelelő adatokat.

Az egyszerűség kedvéért a \lstinline{stack} reprezentációjában az osztály típusát is elmentjük, a két adattag Java reprezentációja a \lstinline{ClassFile} osztályban:
\begin{minted}{java}
public List<Pair<Class<?>, Object>> stack = new ArrayList<>();
public Object[] local = new Object[65535];
\end{minted}
(A \lstinline{Pair} egy egyedi osztály, mely két adattagot tud eltárolni, más nyelvekben \lstinline{tuple}-ként is ismeretes.)

A lokális változók maximális mennyiségét előre tudjuk, ez nem lehet több mint egy 16-bites előjel nélküli szám ($2^{16} = 65536$), alapból ennek az értéke egy 8-bites előjel nélküli szám ($2^8 = 256$) lenne, mivel a \lstinline{store} és \lstinline{load} utasításokat csak egy 8-bites előjel nélküli szám (az \lstinline{index}) követi, viszont a \lstinline{wide} utasítással a \lstinline{store} és \lstinline{load} utasítások módosíthatóak, hogy 2 db 8-bites előjel nélküli számot olvassanak be, tehát lényegében egy 16-bites előjel nélküli számot.

Gyakorlatban ez a szám csökkenthető lenne, tudhatjuk hogy futási időben mennyi lokális változója (illetve a \lstinline{stack} nagyságát is tudhatjuk, tehát tömbként is reprezentálhatnánk) van egy metódusnak. Ez bővebben le van írva a továbbfejlesztési lehetőségekben.

Kényelmi szempontból létezik a \lstinline{CodeIndex} osztály, amely lényegében egy \lstinline{int} szám absztrakciója:
\begin{minted}{java}
class CodeIndex {
    private int index = 0;

    ...
}
\end{minted}
Az absztrakció oka hogy függvényeknek átadva lehessen módosítani ezt a számot; a szám a jelenlegi index a kódot reprezentáló \lstinline{byte} tömbben, megmondja hogy a tömbben lévő melyik indexen levő instrukciót kell végrehajtani.

Az absztrakció különösen észrevehető amikor az \lstinline{if} és \lstinline{goto} utasításokat hajtjuk végre, a \lstinline{ClassFile} objektumunk lokális változója módosítható az \lstinline{Instructions} osztály metódusain keresztül. Mivel a Java érték szerint adja át a paramétereket, ez egy sima \lstinline{int} számmal nem lehetne megoldani.

\subsection{Interpretálás algoritmusa}

Az \ref{alg:ibb}.~algoritmus egy általános elágazás és korlátozás algoritmust (\emph{Branch and Bound algorithm}) mutat be. A \ref{step:selrule}.~lépésben egy megfelelő kiválasztási szabályt kell alkalmazni.
Példa forrása: \href{https://www.inf.u-szeged.hu/actacybernetica/}{Acta Cybernetica (ez egy hiperlink)}.

\begin{algorithm}[H]
\caption{A general interval B\&B algorithm}
\label{alg:ibb}
\textbf{\underline{Funct}} IBB($S,f$)
\begin{algorithmic}[1] % sorszámok megjelenítése minden n. sor előtt, most n = 1
\State Set the working list ${\cal L}_W$ := $\{S\}$ and the final list ${\cal L}_Q$ := $\{\}$
\While{( ${\cal L}_W \neq \emptyset$ )} \label{alg:igoend}
	\State Select an interval $X$ from ${\cal L}_W$ \label{step:selrule}\Comment{Selection rule}
	\State Compute $lbf(X)$ \Comment{Bounding rule}
	\If{$X$ cannot be eliminated} \Comment{Elimination rule}
		\State Divide $X$ into $X^j,\ j=1,\dots, p$, subintervals   \Comment{Division rule}
		\For{$j=1,\ldots,p$}
			\If{$X^j$ satisfies the termination criterion} \Comment{Termination rule}
				\State Store $X^j$ in ${\cal L}_W$
			\Else
				\State Store $X^j$ in ${\cal L}_W$
			\EndIf
		\EndFor
	\EndIf
\EndWhile
\State \textbf{return} ${\cal L}_Q$
\end{algorithmic}
\end{algorithm}

\section{Az interpreter sajátosságai}

\subsection{Erőforrás igények}

A \lstinline{Linux} operációs rendszeren beépített \lstinline{time} programot (illetve a \lstinline{hyperfine} programot) használva az erőforrás igények a tesztfájlokra az alábbiak (a tesztelt számítógép releváns specifikációi: Intel Core i7-8700k processzor 4.7 GHz-en, 16 GB DDR4 memória 2133 MT/s sebességgel):

\begin{table}[htb]
	\centering
	\begin{tabular}{ | l | r | r | r | r | }
		\hline
		\multirow{2}{*}{\textbf{Tesztfájl}} & \multicolumn{2}{ c | }{\textbf{/usr/bin/java}} & \multicolumn{2}{ c | }{\textbf{Jabyinja}} \\
		\cline{2-5}
		& Memória & Futási idő & Memória & Futási idő \\
		\hline \hline		
		Own/Arithmetic.class & 37,1 MB & 21,4 ms & 47,6 MB & 92,8 ms \\
		\hline
		Own/Arrayclass.class & 34,9 MB & 20,9 ms & 54,6 MB & 130,8 ms \\
		\hline
		Own/Arraylist.class & 37,2 MB & 21,4 ms & 49,6 MB & 87,1 ms \\
		\hline
		Own/Athrow.class & 34,6 MB & 20,4 ms & 46,9 MB & 61,7 ms \\
		\hline
		Own/Dup2.class & 36,3 MB & 20,3 ms & 39,2 MB & 45,6 ms \\
		\hline
		Own/Inheritence.class & 34,8 MB & 20,7 ms & 51,9 MB & 98,1 ms \\
		\hline
		Own/Instanceof.class & 38,8 MB & 20,2 ms & 43,1 MB & 64,2 ms \\
		\hline
		Own/Multianewarray.class & 34,4 MB & 20,6 ms & 47,7 MB & 62,9 ms \\
		\hline
		Own/Nested.class & 39,3 MB & 22,1 ms & 47,2 MB & 80,5 ms \\
		\hline
		Own/Ownclass.class & 37,5 MB & 20,2 ms & 61,9 MB & 135,5 ms \\
		\hline
		Own/SwitchAthrow.class & 36,8 MB & 20,5 ms & 40,5 MB & 43,9 ms \\
		\hline
		Own/Template.class & 39,2 MB & 21,2 ms & 51,9 MB & 86,5 ms \\
		\hline
	\end{tabular}
	\caption[Erőforrás különbségek]{A beépített \lstinline{java} és az interpreter közötti erőforrás különbségek}
	\label{jabyinja:resource}
\end{table}

\subsection{A program memóriamodellje}

\subsection{Az önfuttatásról}

\section{Továbbfejlesztési lehetőségek}

\subsection{Invokedynamic utasítás}

Az egyik legszembetűnőbb hiány a szakdolgozatban az egyik nem implementált utasítás, az \lstinline{invokedynamic}, hiánya.
Ez az utasítás számos helyen előfordul Java programokban, leginkább a lambda kifejezésekben (ezen belül is a Konkurens programozás tárgyon megismert \lstinline{Executor} osztály paramétereként), illetve a kiírás során szöveg(ek) és változó(k) konkatenációjánál is ez használt.

Az utóbbi egyszerűen kiküszöbölhető a \lstinline{-XDstringConcat=inline} flag-gel való fordítás során az \lstinline{invokedynamic} utasítás lecserélődik \lstinline{StringBuilder}-en keresztül lévő \lstinline{invokevirtual} és \lstinline{invokespecial} hívásokra.

Az előző viszont sajnos jelen állapotban nem megoldott, és nem is oldható meg egyszerűen. Ahhoz hogy lambda függvények működjenek, az \lstinline{invokedynamic}-ot implementálni kell. Ehhez már az alapvető előkészület megvan, a class fájlban lévő \lstinline{bootstrap} metódosuk egy külön adattag elemeiként el vannak helyezve. A továbbfejlesztés során csak a megfelelő \lstinline{CallSite} helyet, illetve a class fájlban lévő constant pool általi \lstinline{index}-eken levő metódusokkal (illetve paraméterekkel) kell meghívni az éppen leírt függvényt.

\subsection{Java 7 előtti verziók támogatása}

Viszonylag egyszerűen továbbfejleszthető a program hogy Java 7 előtti verzióval fordított class fájlokat is támogasson.

A hiányzó utasítások a \lstinline{ret}, \lstinline{jsr}, \lstinline{jsr_w}, ezek mindegyikéhez csak az szükséges, hogy a megfelelő index-re ugorjunk, a \lstinline{jsr} és \lstinline{jsr_w} utasítások során a visszatérési címet pedig a \lstinline{stack}-re helyezzük.

Természetesen mindegyik utasítás során a megfelelő index-et is be kell olvasnunk a class fájlból, amely a lokális változó megfelelő indexére (\lstinline{ret}), vagy egy adott számot (\lstinline{jsr}, \lstinline{jsr_w}) határoz meg, amely a visszatérési cím, illetve az ugrási cím.

\subsection{Erőforrás igény}

A futási idő táblázata alapján látható hogy a program exponenciálisan lassabb, mint a beépített \lstinline{java} interpretáló program. A program több memóriát is igényel mint szükséges lenne. Ezeknek számos oka is van, ezek közül pár:

\begin{compactitem}
	\item A nem beépített osztályok megfelelő konstruktorait minden egyes alkalommal a program egyesével keresi ki a program. Ez a limitáció nagyon szembetűnő ha sok saját osztállyal dolgozunk. Ilyenkor a futási idő mégjobban lassul
	\item A \lstinline{stack}-nek maximális értéket (65536) foglal a program minden nem beépített függvény meghívása során, viszont a class fájlban ennek a maximális értéke le van írva a megfelelő függvény attribútumaként.
	\item A különböző class fájlok beolvasásának eredménye nincs elmentve, ha egy fájlt be kell olvasnunk, akkor azt minden egyes alkalommal külön-külön megteszünk, ha az eredményt elmentenénk akkor drasztikusan lehetne a sebességen gyorsítani.
\end{compactitem}

\subsection{További tesztelés}

A szakdolgozat írása során megpróbáltam az alapos tesztelésre figyelni, ezért is vannak az alapvető instrukciót egyesével tesztelve (minden tesztfájl-ban külön-külön instrukciók szerepelnek). Viszont a tökéletes program nem létezik, elképzelhető hogy valahol nincs megfelelően a \lstinline{stack} törölve, vagy valamely instrukció mégsem helyes. Tehát a programban elképzelhető a probléma. Ezt a még alaposabb teszteléssel minél inkább meg lehetne cáfolni.

Ehhez egy példa még több tesztfájl mellékelése. A tesztelő környezetbe (Python szkript) viszonylag egyszerűen be lehet helyezni új teszt fájlokat, amely leellenőrzi hogy megfelelő-e a program futása. Továbbfejlesztésként lehet Java programokat írni, majd ezeket a tesztelő környezethez hozzáadni, és ellenőrizni hogy jól fut-e le a program.

\section{Érdekességek a JVM specifikációból}

% \section{JVM Bytecode utasítások}

% Az összes bytecode utasítás implementálva van, ezekről az alábbi táblázatban röviden pár dolog le van írva, nemlegesen a hex kódjuk, extra értékek utánuk, illetve hogy a stack-et hogyan változtatják
% \begin{center}
% 	\begin{longtable}{ | p{0.3\textwidth} | p{0.075\textwidth} | p{0.175\textwidth} | p{0.45\textwidth} | }
		
% 		\hline
% 		\multicolumn{4}{|c|}{\textbf{Bytecode utasítások}}
% 		\\ \hline
		
% 		\emph{Neve} & \emph{HEX} & \emph{Paraméterek} & \emph{Stack}
% 		\\ \hline \hline 
% 		\endfirsthead % első oldal fejléce
		
% 		\hline
% 		\emph{Neve} & \emph{HEX} & \emph{Paraméterek} & \emph{Stack}
% 		\\ \hline \hline
% 		\endhead % többi oldal fejléce
		
%         \emph{NOP}
% 		& 00 & &
% 		\\ \hline

% 		\emph{ACONST\_NULL}
% 		& 01 & & → NULL
% 		\\ \hline
		
% 		\emph{ICONST\_M1}
% 		& 02 & & → -1
% 		\\ \hline
		
% 		\emph{ICONST\_0}
% 		& 03 & & → 0
% 		\\ \hline

%         \emph{ICONST\_1}
% 		& 04 & & → 1
% 		\\ \hline

%         \emph{ICONST\_2}
% 		& 05 & & → 2
% 		\\ \hline

%         \emph{ICONST\_3}
% 		& 06 & & → 3
% 		\\ \hline

%         \emph{ICONST\_4}
% 		& 07 & & → 4
% 		\\ \hline

%         \emph{ICONST\_5}
% 		& 08 & & → 5
% 		\\ \hline

%         \emph{LCONST\_0}
% 		& 09 & & → 0L
% 		\\ \hline

%         \emph{LCONST\_1}
% 		& 0A & & → 1L
% 		\\ \hline
		
%         \emph{FCONST\_0}
% 		& 0B & & → 0.0f
% 		\\ \hline

%         \emph{FCONST\_1}
% 		& 0C & & → 1.0f
% 		\\ \hline

%         \emph{FCONST\_2}
% 		& 0D & & → 2.0f
% 		\\ \hline

%         \emph{DCONST\_0}
% 		& 0E & & → 0.0
% 		\\ \hline

%         \emph{DCONST\_1}
% 		& 0F & & → 1.0
% 		\\ \hline

%         \emph{BIPUSH}
% 		& 10 & \lstinline|u8 value| & → \lstinline|value|
% 		\\ \hline

%         \emph{SIPUSH}
% 		& 11 & \lstinline|u16 value| & → \lstinline|value|
% 		\\ \hline

%         \emph{LDC}
% 		& 12 & \lstinline|u8 index| & → \lstinline|CONSTANT_POOL[index]|
% 		\\ \hline

%         \emph{LDC\_W}
% 		& 13 & \lstinline|u16 index| & → \lstinline|CONSTANT_POOL[index]|
% 		\\ \hline

%         \emph{LDC2\_W}
% 		& 14 & \lstinline|u16 index| & → \lstinline|CONSTANT_POOL[index]|
% 		\\ \hline

%         \emph{ILOAD}
% 		& 15 & \lstinline|u8 index| & → \lstinline|LOCAL[index]|
% 		\\ \hline

%         \emph{LLOAD}
% 		& 16 & \lstinline|u8 index| & → \lstinline|LOCAL[index]|
% 		\\ \hline

%         \emph{FLOAD}
% 		& 17 & \lstinline|u8 index| & → \lstinline|LOCAL[index]|
% 		\\ \hline

%         \emph{DLOAD}
% 		& 18 & \lstinline|u8 index| & → \lstinline|LOCAL[index]|
% 		\\ \hline

%         \emph{ALOAD}
% 		& 19 & \lstinline|u8 index| & → \lstinline|LOCAL[index]|
% 		\\ \hline

%         \emph{ILOAD\_0}
% 		& 1A & & → \lstinline|LOCAL[0]|
% 		\\ \hline

%         \emph{ILOAD\_1}
% 		& 1B & & → \lstinline|LOCAL[1]|
% 		\\ \hline

%         \emph{ILOAD\_2}
% 		& 1C & & → \lstinline|LOCAL[2]|
% 		\\ \hline

%         \emph{ILOAD\_3}
% 		& 1D & & → \lstinline|LOCAL[3]|
% 		\\ \hline
        
%         \emph{LLOAD\_0}
% 		& 1E & & → \lstinline|LOCAL[0]|
% 		\\ \hline

%         \emph{LLOAD\_1}
% 		& 1F & & → \lstinline|LOCAL[1]|
% 		\\ \hline

%         \emph{LLOAD\_2}
% 		& 20 & & → \lstinline|LOCAL[2]|
% 		\\ \hline

%         \emph{LLOAD\_3}
% 		& 21 & & → \lstinline|LOCAL[3]|
% 		\\ \hline

%         \emph{FLOAD\_0}
% 		& 22 & & → \lstinline|LOCAL[0]|
% 		\\ \hline

%         \emph{FLOAD\_1}
% 		& 23 & & → \lstinline|LOCAL[1]|
% 		\\ \hline

%         \emph{FLOAD\_2}
% 		& 24 & & → \lstinline|LOCAL[2]|
% 		\\ \hline

%         \emph{FLOAD\_3}
% 		& 25 & & → \lstinline|LOCAL[3]|
% 		\\ \hline
        
%         \emph{DLOAD\_0}
% 		& 26 & & → \lstinline|LOCAL[0]|
% 		\\ \hline

%         \emph{DLOAD\_1}
% 		& 27 & & → \lstinline|LOCAL[1]|
% 		\\ \hline

%         \emph{DLOAD\_2}
% 		& 28 & & → \lstinline|LOCAL[2]|
% 		\\ \hline

%         \emph{DLOAD\_3}
% 		& 29 & & → \lstinline|LOCAL[3]|
% 		\\ \hline
        
%         \emph{ALOAD\_0}
% 		& 2A & & → \lstinline|LOCAL[0]|
% 		\\ \hline

%         \emph{ALOAD\_1}
% 		& 2B & & → \lstinline|LOCAL[1]|
% 		\\ \hline

%         \emph{ALOAD\_2}
% 		& 2C & & → \lstinline|LOCAL[2]|
% 		\\ \hline

%         \emph{ALOAD\_3}
% 		& 2D & & → \lstinline|LOCAL[3]|
% 		\\ \hline
        
%         \emph{IALOAD}
% 		& 2E & & \lstinline|arrayref, index| → \lstinline|value|
% 		\\ \hline

%         \emph{LALOAD}
% 		& 2F & & \lstinline|arrayref, index| → \lstinline|value|
% 		\\ \hline

%         \emph{FALOAD}
% 		& 30 & & \lstinline|arrayref, index| → \lstinline|value|
% 		\\ \hline

%         \emph{DALOAD}
% 		& 31 & & \lstinline|arrayref, index| → \lstinline|value|
% 		\\ \hline
        
%         \emph{AALOAD}
% 		& 32 & & \lstinline|arrayref, index| → \lstinline|value|
% 		\\ \hline

%         \emph{BALOAD}
% 		& 33 & & \lstinline|arrayref, index| → \lstinline|value|
% 		\\ \hline

%         \emph{CALOAD}
% 		& 34 & & \lstinline|arrayref, index| → \lstinline|value|
% 		\\ \hline

%         \emph{SALOAD}
% 		& 35 & & \lstinline|arrayref, index| → \lstinline|value|
% 		\\ \hline

%         \emph{ISTORE}
% 		& 36 & \lstinline|u8 index| & \lstinline|value| →
% 		\\ \hline

%         \emph{LSTORE}
% 		& 37 & \lstinline|u8 index| & \lstinline|value| →
% 		\\ \hline

%         \emph{FSTORE}
% 		& 38 & \lstinline|u8 index| & \lstinline|value| →
% 		\\ \hline

%         \emph{DSTORE}
% 		& 39 & \lstinline|u8 index| & \lstinline|value| →
% 		\\ \hline
        
%         \emph{ASTORE}
% 		& 3A & \lstinline|u8 index| & \lstinline|objectref| →
% 		\\ \hline

%         \emph{ISTORE\_0}
% 		& 3B & & \lstinline|value| →
% 		\\ \hline

%         \emph{ISTORE\_1}
% 		& 3C & & \lstinline|value| →
% 		\\ \hline

%         \emph{ISTORE\_2}
% 		& 3D & & \lstinline|value| →
% 		\\ \hline

%         \emph{ISTORE\_3}
% 		& 3E & & \lstinline|value| →
% 		\\ \hline

%         \emph{LSTORE\_0}
% 		& 3F & & \lstinline|value| →
% 		\\ \hline

%         \emph{LSTORE\_1}
% 		& 40 & & \lstinline|value| →
% 		\\ \hline

%         \emph{LSTORE\_2}
% 		& 41 & & \lstinline|value| →
% 		\\ \hline

%         \emph{LSTORE\_3}
% 		& 42 & & \lstinline|value| →
% 		\\ \hline

%         \emph{FSTORE\_0}
% 		& 43 & & \lstinline|value| →
% 		\\ \hline

%         \emph{FSTORE\_1}
% 		& 44 & & \lstinline|value| →
% 		\\ \hline

%         \emph{FSTORE\_2}
% 		& 45 & & \lstinline|value| →
% 		\\ \hline

%         \emph{FSTORE\_3}
% 		& 46 & & \lstinline|value| →
% 		\\ \hline

%         \emph{DSTORE\_0}
% 		& 47 & & \lstinline|value| →
% 		\\ \hline

%         \emph{DSTORE\_1}
% 		& 48 & & \lstinline|value| →
% 		\\ \hline

%         \emph{DSTORE\_2}
% 		& 49 & & \lstinline|value| →
% 		\\ \hline

%         \emph{DSTORE\_3}
% 		& 4A & & \lstinline|value| →
% 		\\ \hline

%         \emph{ASTORE\_0}
% 		& 4B & & \lstinline|value| →
% 		\\ \hline

%         \emph{ASTORE\_1}
% 		& 4C & & \lstinline|value| →
% 		\\ \hline

%         \emph{ASTORE\_2}
% 		& 4D & & \lstinline|value| →
% 		\\ \hline

%         \emph{ASTORE\_3}
% 		& 4E & & \lstinline|value| →
% 		\\ \hline

%         \emph{IASTORE}
% 		& 4F & & \lstinline|arrayref, index, value| →
% 		\\ \hline

%         \emph{LASTORE}
% 		& 50 & & \lstinline|arrayref, index, value| →
% 		\\ \hline

%         \emph{FASTORE}
% 		& 51 & & \lstinline|arrayref, index, value| →
% 		\\ \hline

%         \emph{DASTORE}
% 		& 52 & & \lstinline|arrayref, index, value| →
% 		\\ \hline

%         \emph{AASTORE}
% 		& 53 & & \lstinline|arrayref, index, value| →
% 		\\ \hline

%         \emph{BASTORE}
% 		& 54 & & \lstinline|arrayref, index, value| →
% 		\\ \hline

%         \emph{CASTORE}
% 		& 55 & & \lstinline|arrayref, index, value| →
% 		\\ \hline

%         \emph{SASTORE}
% 		& 56 & & \lstinline|arrayref, index, value| →
% 		\\ \hline

%         \emph{POP}
% 		& 57 & & \lstinline|value| →
% 		\\ \hline

%         \emph{POP2}
% 		& 58 & & \lstinline|{value2, value1}| →
% 		\\ \hline

%         \emph{DUP}
% 		& 59 & & \lstinline|value| → \lstinline|value, value|
% 		\\ \hline

%         \emph{DUP\_X1}
% 		& 5A & & \lstinline|{value2, value1}| → \\
%         & & & \lstinline|value1, {value2, value1}|
% 		\\ \hline

%         \emph{DUP\_X2}
% 		& 5B & & \lstinline|value3, value2, value1| → \\
%         & & & \lstinline|value1, value3, value2, value1|
% 		\\ \hline

%         \emph{DUP2}
% 		& 5C & & \lstinline|{value2, value1}| → \\
%         & & & \lstinline|{value2, value1}, {value2, value1}|
% 		\\ \hline

%         \emph{DUP2\_X1}
% 		& 5D & & \lstinline|value3, {value2, value1}| → \\
%         & & & \lstinline|{value2, value1}, value3, {value2, value1}|
% 		\\ \hline

%         \emph{DUP2\_X2}
% 		& 5E & & \lstinline|{value4, value3}, {value2, value1}| → \\
%         & & & \lstinline|{value2, value1}, {value4, value3}, {value2, value1}|
% 		\\ \hline

%         \emph{SWAP}
% 		& 5F & & \lstinline|value2, value1| → \lstinline|value1, value2|
% 		\\ \hline
        
%         \emph{IADD}
% 		& 60 & & \lstinline|value1, value2| → \lstinline|result|
% 		\\ \hline

%         \emph{LADD}
% 		& 61 & & \lstinline|value1, value2| → \lstinline|result|
% 		\\ \hline

%         \emph{FADD}
% 		& 62 & & \lstinline|value1, value2| → \lstinline|result|
% 		\\ \hline

%         \emph{DADD}
% 		& 63 & & \lstinline|value1, value2| → \lstinline|result|
% 		\\ \hline
        
%         \emph{ISUB}
% 		& 64 & & \lstinline|value1, value2| → \lstinline|result|
% 		\\ \hline

%         \emph{LSUB}
% 		& 65 & & \lstinline|value1, value2| → \lstinline|result|
% 		\\ \hline

%         \emph{FSUB}
% 		& 66 & & \lstinline|value1, value2| → \lstinline|result|
% 		\\ \hline

%         \emph{DSUB}
% 		& 67 & & \lstinline|value1, value2| → \lstinline|result|
% 		\\ \hline
        
%         \emph{IMUL}
% 		& 68 & & \lstinline|value1, value2| → \lstinline|result|
% 		\\ \hline

%         \emph{LMUL}
% 		& 69 & & \lstinline|value1, value2| → \lstinline|result|
% 		\\ \hline

%         \emph{FMUL}
% 		& 6A & & \lstinline|value1, value2| → \lstinline|result|
% 		\\ \hline

%         \emph{DMUL}
% 		& 6B & & \lstinline|value1, value2| → \lstinline|result|
% 		\\ \hline
        
%         \emph{IDIV}
% 		& 6C & & \lstinline|value1, value2| → \lstinline|result|
% 		\\ \hline

%         \emph{LDIV}
% 		& 6D & & \lstinline|value1, value2| → \lstinline|result|
% 		\\ \hline

%         \emph{FDIV}
% 		& 6E & & \lstinline|value1, value2| → \lstinline|result|
% 		\\ \hline

%         \emph{DDIV}
% 		& 6F & & \lstinline|value1, value2| → \lstinline|result|
% 		\\ \hline
        
%         \emph{IREM}
% 		& 70 & & \lstinline|value1, value2| → \lstinline|result|
% 		\\ \hline

%         \emph{LREM}
% 		& 71 & & \lstinline|value1, value2| → \lstinline|result|
% 		\\ \hline

%         \emph{FREM}
% 		& 72 & & \lstinline|value1, value2| → \lstinline|result|
% 		\\ \hline

%         \emph{DREM}
% 		& 73 & & \lstinline|value1, value2| → \lstinline|result|
% 		\\ \hline
        
%         \emph{INEG}
% 		& 74 & & \lstinline|value| → \lstinline|result|
% 		\\ \hline

%         \emph{LNEG}
% 		& 75 & & \lstinline|value| → \lstinline|result|
% 		\\ \hline

%         \emph{FNEG}
% 		& 76 & & \lstinline|value| → \lstinline|result|
% 		\\ \hline

%         \emph{DNEG}
% 		& 77 & & \lstinline|value| → \lstinline|result|
% 		\\ \hline

%         \emph{ISHL}
% 		& 78 & & \lstinline|value1, value2| → \lstinline|result|
% 		\\ \hline

%         \emph{LSHL}
% 		& 79 & & \lstinline|value1, value2| → \lstinline|result|
% 		\\ \hline

%         \emph{ISHR}
% 		& 7A & & \lstinline|value1, value2| → \lstinline|result|
% 		\\ \hline

%         \emph{LSHR}
% 		& 7B & & \lstinline|value1, value2| → \lstinline|result|
% 		\\ \hline

%         \emph{IUSHR}
% 		& 7C & & \lstinline|value1, value2| → \lstinline|result|
% 		\\ \hline

%         \emph{LUSHR}
% 		& 7D & & \lstinline|value1, value2| → \lstinline|result|
% 		\\ \hline

%         \emph{IAND}
% 		& 7E & & \lstinline|value1, value2| → \lstinline|result|
% 		\\ \hline

%         \emph{LAND}
% 		& 7F & & \lstinline|value1, value2| → \lstinline|result|
% 		\\ \hline

%         \emph{IOR}
% 		& 80 & & \lstinline|value1, value2| → \lstinline|result|
% 		\\ \hline

%         \emph{LOR}
% 		& 81 & & \lstinline|value1, value2| → \lstinline|result|
% 		\\ \hline

%         \emph{IXOR}
% 		& 82 & & \lstinline|value1, value2| → \lstinline|result|
% 		\\ \hline

%         \emph{LXOR}
% 		& 83 & & \lstinline|value1, value2| → \lstinline|result|
% 		\\ \hline

%         \emph{IINC}
% 		& 84 & \lstinline|u8 index, | & \lstinline|..| → \lstinline|..| \\
%         & & \lstinline|u8 const| &
% 		\\ \hline

%         \emph{I2L}
% 		& 85 & & \lstinline|value| → \lstinline|result|
% 		\\ \hline

%         \emph{I2F}
% 		& 86 & & \lstinline|value| → \lstinline|result|
% 		\\ \hline

%         \emph{I2D}
% 		& 87 & & \lstinline|value| → \lstinline|result|
% 		\\ \hline

%         \emph{L2I}
% 		& 88 & & \lstinline|value| → \lstinline|result|
% 		\\ \hline

%         \emph{L2F}
% 		& 89 & & \lstinline|value| → \lstinline|result|
% 		\\ \hline

%         \emph{L2D}
% 		& 8A & & \lstinline|value| → \lstinline|result|
% 		\\ \hline

%         \emph{F2I}
% 		& 8B & & \lstinline|value| → \lstinline|result|
% 		\\ \hline

%         \emph{F2L}
% 		& 8C & & \lstinline|value| → \lstinline|result|
% 		\\ \hline

%         \emph{F2D}
% 		& 8D & & \lstinline|value| → \lstinline|result|
% 		\\ \hline

%         \emph{D2I}
% 		& 8E & & \lstinline|value| → \lstinline|result|
% 		\\ \hline

%         \emph{D2L}
% 		& 8F & & \lstinline|value| → \lstinline|result|
% 		\\ \hline

%         \emph{D2F}
% 		& 90 & & \lstinline|value| → \lstinline|result|
% 		\\ \hline

%         \emph{I2B}
% 		& 91 & & \lstinline|value| → \lstinline|result|
% 		\\ \hline

%         \emph{I2C}
% 		& 92 & & \lstinline|value| → \lstinline|result|
% 		\\ \hline

%         \emph{I2S}
% 		& 93 & & \lstinline|value| → \lstinline|result|
% 		\\ \hline

%         \emph{LCMP}
% 		& 94 & & \lstinline|value1, value2| → \lstinline|result|
% 		\\ \hline

%         \emph{FCMPL}
% 		& 95 & & \lstinline|value1, value2| → \lstinline|result|
% 		\\ \hline

%         \emph{FCMPG}
% 		& 96 & & \lstinline|value1, value2| → \lstinline|result|
% 		\\ \hline

%         \emph{DCMPL}
% 		& 97 & & \lstinline|value1, value2| → \lstinline|result|
% 		\\ \hline

%         \emph{DCMPG}
% 		& 98 & & \lstinline|value1, value2| → \lstinline|result|
% 		\\ \hline

%         \emph{IFEQ}
% 		& 99 & \lstinline|u16 branch| & \lstinline|value| →
% 		\\ \hline

%         \emph{IFNE}
% 		& 9A & \lstinline|u16 branch| & \lstinline|value| →
% 		\\ \hline

%         \emph{IFLT}
% 		& 9B & \lstinline|u16 branch| & \lstinline|value| →
% 		\\ \hline

%         \emph{IFGE}
% 		& 9C & \lstinline|u16 branch| & \lstinline|value| →
% 		\\ \hline

%         \emph{IFGT}
% 		& 9D & \lstinline|u16 branch| & \lstinline|value| →
% 		\\ \hline

%         \emph{IFLE}
% 		& 9E & \lstinline|u16 branch| & \lstinline|value| →
% 		\\ \hline

%         \emph{IF\_ICMPEQ}
% 		& 9F & \lstinline|u16 branch| & \lstinline|value1, value2| →
% 		\\ \hline

%         \emph{IF\_ICMPNE}
% 		& A0 & \lstinline|u16 branch| & \lstinline|value1, value2| →
% 		\\ \hline

%         \emph{IF\_ICMPLT}
% 		& A1 & \lstinline|u16 branch| & \lstinline|value1, value2| →
% 		\\ \hline

%         \emph{IF\_ICMPGE}
% 		& A2 & \lstinline|u16 branch| & \lstinline|value1, value2| →
% 		\\ \hline

%         \emph{IF\_ICMPGT}
% 		& A3 & \lstinline|u16 branch| & \lstinline|value1, value2| →
% 		\\ \hline

%         \emph{IF\_ICMPLE}
% 		& A4 & \lstinline|u16 branch| & \lstinline|value1, value2| →
% 		\\ \hline

%         \emph{IF\_ACMPEQ}
% 		& A5 & \lstinline|u16 branch| & \lstinline|value1, value2| →
% 		\\ \hline

%         \emph{IF\_ACMPNE}
% 		& A6 & \lstinline|u16 branch| & \lstinline|value1, value2| →
% 		\\ \hline

%         \emph{GOTO}
% 		& A7 & \lstinline|u16 branch| & \lstinline|..| → \lstinline|..|
% 		\\ \hline

%         \emph{JSR}
% 		& A8 & \lstinline|u16 branch| & → \lstinline|address|
% 		\\ \hline

%         \emph{RET}
% 		& A9 & \lstinline|u8 index| & \lstinline|..| → \lstinline|..|
% 		\\ \hline

%         \emph{TABLESWITCH}
% 		& AA & \lstinline|[0-3] byte padding, | & \\
%         & & \lstinline|u32 default, | & \\
%         & & \lstinline|u32 low, | & \lstinline|index| → \\
%         & & \lstinline|u32 high, | & \\
%         & & \lstinline|jump offsets| & 
% 		\\ \hline

%         \emph{LOOKUPSWITCH}
% 		& AB & \lstinline|[0-3] byte padding, | & \\
%         & & \lstinline|u32 default, | & \\
%         & & \lstinline|u32 npairs, | & \lstinline|key| → \\
%         & & \lstinline|match-offset pairs| & 
% 		\\ \hline

%         \emph{IRETURN}
% 		& AC & & \lstinline|value| →
% 		\\ \hline

%         \emph{LRETURN}
% 		& AD & & \lstinline|value| →
% 		\\ \hline

%         \emph{FRETURN}
% 		& AE & & \lstinline|value| →
% 		\\ \hline

%         \emph{DRETURN}
% 		& AF & & \lstinline|value| →
% 		\\ \hline

%         \emph{ARETURN}
% 		& B0 & & \lstinline|objectref| →
% 		\\ \hline

%         \emph{RETURN}
% 		& B1 & & \lstinline|..| →
% 		\\ \hline

%         \emph{GETSTATIC}
% 		& B2 & \lstinline|u16 index| & → \lstinline|value|
% 		\\ \hline

%         \emph{PUTSTATIC}
% 		& B3 & \lstinline|u16 index| & \lstinline|value| → 
% 		\\ \hline

%         \emph{GETFIELD}
% 		& B4 & \lstinline|u16 index| & \lstinline|objectref| → \lstinline|value|
% 		\\ \hline

%         \emph{PUTFLIED}
% 		& B5 & \lstinline|u16 index| & \lstinline|objectref, value| → 
% 		\\ \hline

%         \emph{INVOKEVIRTUAL}
% 		& B6 & \lstinline|u16 index| & \lstinline|objectref, [args]| → \lstinline|result|
% 		\\ \hline

%         \emph{INVOKESPECIAL}
% 		& B7 & \lstinline|u16 index| & \lstinline|objectref, [args]| → \lstinline|result|
% 		\\ \hline

%         \emph{INVOKESTATIC}
% 		& B8 & \lstinline|u16 index| & \lstinline|[args]| → \lstinline|result|
% 		\\ \hline

%         \emph{INVOKEINTERFACE}
% 		& B9 & \lstinline|u16 index| & \lstinline|objectref, [args]| → \lstinline|result| \\
%         & & \lstinline|u8 count, 0| &
% 		\\ \hline

%         \emph{INVOKEDYNAMIC}
% 		& BA & \lstinline|u16 index, | & \lstinline|[args]| → \lstinline|result| \\
%         & & \lstinline|0, 0| &
% 		\\ \hline

%         \emph{NEW}
% 		& BB & \lstinline|u16 index| & → \lstinline|objectref|
% 		\\ \hline

%         \emph{NEWARRAY}
% 		& BC & \lstinline|u8 atype| & \lstinline|count| → \lstinline|arrayref|
% 		\\ \hline

%         \emph{ANEWARRAY}
% 		& BD & \lstinline|u16 index| & \lstinline|count| → \lstinline|arrayref|
% 		\\ \hline

%         \emph{ARRAYLENGTH}
% 		& BE & \lstinline|u16 index| & \lstinline|arrayref| → \lstinline|length|
% 		\\ \hline

%         \emph{ATHROW}
% 		& BF & & \lstinline|objectref| → \lstinline|objectref|
% 		\\ \hline

%         \emph{CHECKCAST}
% 		& C0 & \lstinline|u16 index| & \lstinline|objectref| → \lstinline|objectref|
% 		\\ \hline

%         \emph{INSTANCEOF}
% 		& C1 & \lstinline|u16 index| & \lstinline|objectref| → \lstinline|result|
% 		\\ \hline

%         \emph{MONITORENTER}
% 		& C2 & \lstinline|u16 index| & \lstinline|objectref| →
% 		\\ \hline

%         \emph{MONITOREXIT}
% 		& C3 & \lstinline|u16 index| & \lstinline|objectref| →
% 		\\ \hline

%         \emph{WIDE}
% 		& C4 & \lstinline|u8 opcode| & \\
%         & & \lstinline|u16 index| & \\
%         & & \lstinline|vagy| & \lstinline|mint az adott utasitas| \\
%         & & \lstinline|u8 opcode,| & \\
%         & & \lstinline|u16 index,| & \\
%         & & \lstinline|u16 count| & 
% 		\\ \hline

%         \emph{MULTIANEWARRAY}
% 		& C5 & \lstinline|u16 index, | & \lstinline|[counts]| → \lstinline|arrayref| \\
%         & & \lstinline|u8 dimensions| & 
% 		\\ \hline

%         \emph{IFNULL}
% 		& C6 & \lstinline|u16 branch| & \lstinline|value| →
% 		\\ \hline

%         \emph{IFNONNULL}
% 		& C7 & \lstinline|u16 branch| & \lstinline|value| →
% 		\\ \hline

%         \emph{GOTO\_W}
% 		& C8 & \lstinline|u32 branch| & \lstinline|..| → \lstinline|..|
% 		\\ \hline
        
%         \emph{JSR\_W}
% 		& C9 & \lstinline|u32 branch| & → \lstinline|address|
% 		\\ \hline
		
% 		\caption{A JVM Bytecode különböző utasításai}
% 		\label{tab:example-3}		
% 	\end{longtable}
% \end{center}